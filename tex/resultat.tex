\section{Resultat}
Från det historiska perspektivet på människors uppfinningar och ändringar kan nu de tekniska faktorerna sammanställas. 

\subsection{Vapen och redskap}
Med utvecklandet av nya vapen och redskap har djur kunnat slaktas allt mer effektivt. Från uppfinnandet av de enkla (men grundläggande) vapnen och redskapen i sten, förbättringen av dessa med nyttjandet av metall och nya metoder för att döda med elektrifiering och gasning. 

\subsection{Boskapsskötsel}
Avling av djur började nyttjas efter istiden för ut mer av köttproduktionen och med penicillinet kunde djuren hållas friska och växa snabbare.  

\subsection{Konservering}
Konserveringen av kött har varit under utveckling sedan stenåldern. Först med soltorkning och rökning, under antiken salt och till nutidens konservburkar och den elektroniska frysen.   

\subsection{Maskiner, fabriker och massproducering}
Det medeltida vattenljulet ledde vägen till effektivare maskiner och ett mekaniserat tänkesätt. Ångmaskinen kom att revolutionera fabriksdriften för att sedan ersättas av elektriteten och nya motorer. 
\newline
\newline
Nya tankesätt som organiserade arbetskraften i fabrikerna under 1700-talet ledde till en effektivare industri. Löpandebandets introducering möjliggjorde massproducering på en helt annan skala. 

\subsection{Transport och kommunikation}
Fartygen som utvecklades från antiken var 
