\section{Teori}

\subsection{Jägarmänniskan}
stenålder, redskap och redskapstillverkning, fångsteranordningar/fällor, skivtekniken, sågar, spånredskap, jägare, befolkningsökning pga kött, 

\subsection{Bofasthet och boskapsskötsel}
efter istiden, minskad tillgång på storvilt pga för mycket jakt och klimatförändringar, jordbruk och bofasthet, boskapsskötsel som dominerande näringar, livsmedelsproduktion, domesticering av husdjur, befolkningsökning, Jeriko, handelsutbyte. Allt mer krävande organiserade samarbeten (bevattningsproblem) krävde någon form av hierarki för att styras, fastare samhällsapparat.   

