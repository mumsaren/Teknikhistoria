\section{Teori}
Nedan följer utvecklingen av köttindustrin från stenåldern fram tills idag.

\subsection{Jägarmänniskan}
Under stenåldern var människans främsta födokälla storvilt och för att kunna jaga dessa ställdes krav på människans redskap och verktyg. Tillverkningsmetoder utvecklades för att ge mer effektiva och dödliga vapen med skarpare egg. Olika fångsteranordningar och fällor för att fånga och döda djuren utvecklades också. Med skivtekniken, där avslagen används, kunde redskap med mycket god skärpa tillverkas som sågliknande redskap. Under senare tid tillkom även mer avancerade vapen som spjut och yxor.  
\newline
\newline
Människorna hade även god kännedom om djur som får, getter, hästar och kor vilket tyder på en tidig grund för boskapsskötsel. 
\newline
(referera Den skapande människan)
     
\subsection{Bofasthet och boskapsskötsel}
Efter istiden blev det en minskad tillgång på storvilt, dels på grund av klimatförändringarna som följde, dels på grund av människans effektiva jakt. Människan blev nu tvungen att slå sig ned och skapa mer permanenta bosättningar. De främsta näringstillgångarna blev nu istället livsmedelsproduktion från jordbruk och boskapskötsel. Domesticering (avling) av får, getter, nötboskap och grisar för att bland annat få en större köttproduktion började ske under denna period. 
\newline
\newline
Med en växande betydelse för boskapsskötsel och ett tilltagande befolkningstryck medförde sökande efter nya betesmarker. Kring floderna Eufrat och Tigris finns naturliga betesområden och det var här civilisationens vagga Mesopotamien uppkom.   
\newline
\newline
(referera Den skapande människan)

\subsection{Metall}


