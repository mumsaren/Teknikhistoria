\section{Historiskt perspektiv}
Nedan följer viktiga uppfinningar och ändringar människor gjort från stenåldern fram tills idag. Dessa fakta kommer jag referera till i min rapport. 
 
\subsection{Stenåldern}
Under stenåldern var människans främsta födokälla storvilt och för att kunna jaga dessa ställdes krav på människans redskap och verktyg. Tillverkningsmetoder utvecklades för att ge mer effektiva och dödliga vapen med skarpare egg. Olika fångsteranordningar och fällor för att fånga och döda djuren utvecklades också. Med skivtekniken, där avslagen används, kunde redskap med mycket god skärpa tillverkas som sågliknande redskap. Under senare delen av stenåldern tillkom även mer avancerade vapen som spjut och yxor. \citep{denskapande} 
\newline
\newline
Människorna hade under denna tidsepok god kännedom om djur som får, getter, hästar och kor vilket tyder på en tidig grund för boskapsskötsel. \citep{denskapande}
     
\subsection{Efter istiden}
Efter istiden blev det en minskad tillgång på storvilt, dels på grund av klimatförändringarna som följde, dels på grund av människans effektiva jakt. Människan blev nu tvungen att slå sig ned och skapa mer permanenta bosättningar. De främsta näringstillgångarna blev nu istället livsmedelsproduktion från jordbruk och boskapskötsel. För att få en ökad köttproduktion nyttjade de tidiga boskapsskötarna domesticering (avling) av får, getter, nötboskap och grisar. \citep{denskapande}
\newline
\newline
Den växande betydelsen för boskapsskötsel i kombination med ett tilltagande befolkningstryck medförde ett behov at hitta nya betesmarker. Kring floderna Eufrat och Tigris finns naturliga betesområden och det var här civilisationens vagga Mesopotamien senare uppkom. \citep{denskapande}   

\subsection{Antiken}
Med ugnar kunde metallen nyttjas av människan, främst koppar och guld samt legeringar som brons. Bronset ersattes senare av järn med gjutjärnstekniken. Vapen och verktyg kunde nu förstärkas och effektiviseras. Senare effektiviserades ugnen till smältugnar. Med allt djupare gruvor uppfanns primitiva pumpar för att få upp grundvattnet som orsakade översvämningar. \citep{denskapande}
\newline
\newline 
Det nya samhället var präglat av fast boende ställdes krav på lagring och konservering. \citep{denskapande}
\newline
\newline
Gamla konserveringsmetoder för kött var antingen rökning eller soltorkning men nu började även saltet användas som konserveringsmetod. Saltet var så viktigt att produktionen kontrollerades av härskarna. 
(referera the story of food preservation) 
\newline
\newline
Saltet blev en mycket viktig handelsvara. I antikens Kina utvecklades en ny effektiv metod för att utvinna salt för att täcka det ökande behovet - djupborrningstekniken. Det var för dåtiden en mycket modern teknik som inte skulle bli känd i Europa förrän långt senare. \citep{denskapande}
\newline
\newline
Det var långt ifrån alla regioner som kunde klara av sin egen försörjning och med en ökande handel ställde detta krav på transporterna. Under antiken fanns det stora fartyg som kunde frakta hundratals ton och med uppfinningen av vattenfast murbruk kunde stora hamnanläggningar till dessa konstrueras. Vägnätet utvecklades också men spelade en förhållandevis liten roll då dåtidens vagnar inte kunte transportera några större laster. \citep{denskapande}

\subsection{Medeltiden}    
Under medeltiden lyckades människan ta naturens krafter i anspråk genom en ny uppfinning - vattenkvarnen. Dessa kunde användas för att göra det mesta, till exempel driva sågverk och maskiner. Under denna tid tenderande människan att börja separeras från kraftutövningen. \citep{medeltiden}
\newline
\newline
Segelfartygen förbättrades som följd av den ökande handelskonkurrensen och kampen om erövring mellan nationer. Detta ledde till längre och ökade transporter. \citep{denskapande}

\subsection{1700-talet}
På 1700-talet ökade intresset för avel, ej baserat på någon teoretisk kunskap utan på praktisk, vilket ledde till att man fick fram större och kraftigare slaktdjur. Förbättrad utfodring var också en bidragande faktor till att djuren växte snabbare. \citep{denskapande}
\newline
\newline
En av de tidigaste mekaniserade fabrikerna byggdes i England som nyttjade maskiner och arbetare på ett organiserat sätt. Ett inplanterade av en ny arbetsdisciplin och organisation där arbetarna underordnade sig fabrikslivets villkor ledde vägen till en effektivare industri. \citep{denskapande}
\newline
\newline
Med en ökande befolkningsmängd och urbanisering ökade efterfrågan på kol som användes vid järnproduktionen. Allt djupare gruvor krävdes och de översvämmades ofta av grundvattnet. Detta ställde i sin tur krav på bättre pumpar och lösningen blev då ångmaskinen. \citep{industriell}
Ångmaskinen revolutionerade fabriksdriften och banade väg för förbättrade och nya transportmedel som ångfartyg och ånglok. \citep{denskapande}

\subsection{1800-talet}
Med ångloket och järnvägens uppkomst under 1800-talet kunde råvor och färdiga produkter fraktas i helt andra mängder än tidigare till lands och förbättrade dessutom kommunikationen mellan samhällena. \citep{denskapande}
\newline
\newline
Under 1800-talet tillkom nya tekniker för konservering av mat. Konserveringsburkar, frysar kylda med is och sedan den elektroniskt kompressordrivna frysen möjliggjorde längre förvaring av mat. \citep{denskapande} 
Det möjliggjorde också en exportering av mat på en global skala. \citep{edgerton}
\newline
\newline
En viktig förutsättning för massproduktion var löpandebandmetoden som gick ut på att varan förs fram på ett kontinuerligt transportband och varje arbetare utför endast ett visst arbetsmoment. Detta ledde till ett starkt minskat behov av arbetskraft och att den var lätt utbytbar - faktorer som möjliggjorde effektivare, massproducerande företag. \citep{denskapande}
\newline
\newline
Elektricitetens införande i samhället och nya revolutionerande motorer var bidragande faktorer till en allt rationellare fabriksdrift. \cite{denskapande}
\newline
\newline
I slutet på 1800-talet upptäcktes att det gick att avrätta människor med elektrifiering. Det innebar att offret fick stark ström genom kroppen. Senare kom denna avrättningsmetod att tillämpas även inom slaktindustrin. \citep{edgerton}

\subsection{1900-talet tills idag}
Henry Fords massproducerade bensindrivna bil var revolutionerande för samhällsutvecklingen. Bilarna effektiviserade transporter och var således betydande för exempelvis köttindustrin. \citep{denskapande}
\newline
\newline
Penicillin utveklades under 1940-talet och användes för att döda bakterier i människor. Det blev även en viktig del i kontrollerandet av sjukdomar i de packade djurpopulationerna i den nya industrialiserade djurhållningen. Det upptäcktes även att kycklingar växte snabbare av penicillin vilket ledde till att det tillsattes i djurens föda. \citep{denskapande}
\newline
\newline
Under 1900-talet började dödning med gas användas, först på människor i avrättningssyfte men detta kom senare att påverka hur grisar slaktades \citep{denskapande}
\newline
\newline
Informationssamhällets framväxt med radio, telefon, tv och senare internet utvecklade kommunikationen och hjälpte till marknaden för företagen. \citep{kommunikation}