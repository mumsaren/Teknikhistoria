\section{Teori}
Nedan följer utvecklingen av köttindustrin från stenåldern fram tills idag.

\subsection{Jägarmänniskan}
Under stenåldern var människans främsta födokälla storvilt och för att kunna jaga dessa ställdes krav på människans redskap och verktyg. Tillverkningsmetoder utvecklades för att ge mer effektiva och dödliga vapen med skarpare egg. Olika fångsteranordningar och fällor för att fånga och döda djuren utvecklades också. Med skivtekniken, där avslagen används, kunde redskap med mycket god skärpa tillverkas som sågliknande redskap. Under senare tid tillkom även mer avancerade vapen som spjut och yxor.  
\newline
\newline
Människorna hade även god kännedom om djur som får, getter, hästar och kor vilket tyder på en tidig grund för boskapsskötsel. 
\newline
(referera Den skapande människan)
     
\subsection{Bofasthet och boskapsskötsel}
Efter istiden blev det en minskad tillgång på storvilt, dels på grund av klimatförändringarna som följde, dels på grund av människans effektiva jakt. Människan blev nu tvungen att slå sig ned och skapa mer permanenta bosättningar. De främsta näringstillgångarna blev nu istället livsmedelsproduktion från jordbruk och boskapskötsel. Domesticering (avling) av får, getter, nötboskap och grisar för att bland annat få en större köttproduktion började ske under denna period. 
\newline
\newline
Med en växande betydelse för boskapsskötsel och ett tilltagande befolkningstryck medförde sökande efter nya betesmarker. Kring floderna Eufrat och Tigris finns naturliga betesområden och det var här civilisationens vagga Mesopotamien uppkom.   
\newline
\newline
(referera Den skapande människan)

\subsection{Metaller och lagringsbehov}
När metallen började nyttjas av människan, då främst koppar och guld och senare legeringar som brons, kunde effektivare vapen tillverkas.   
\newline
\newline 
Då det nya samhället var präglat av fast boende ställdes krav på lagring och konservering. (referera Den skapande människan) 
\newline
\newline
Gamla konserveringsmetoder för kött var antingen rökning eller soltorkning men nu började även saltet användas som konserveringsmetod. Saltet var så viktigt att produktionen kontrollerades av härskarna. 
(referera the story of food preservation) 
\newline
\newline
Saltet blev en mycket viktig handelsvara. I antikens Kina utvecklades en ny effektiv metod för att utvinna salt för att täcka det ökande behovet - djupborrningstekniken. Det var för dåtiden en mycket modern teknik som inte skulle bli känd i Europa förrän långt senare. (referera Den skapande människan)

\subsection{Transport under antiken}
Det var långt ifrån alla regioner som kunde klara av sin egen försörjning och med en ökande handel ställde detta krav på transporterna. Under antiken fanns det stora fartyg som kunde frakta hundratals ton och med uppfinningen av vattenfast murbruk kunde stora hamnanläggningar till dessa konstrueras. Vägnätet utvecklades också men spelade en förhållandevis liten roll då dåtidens vagnar inte kunte transportera några större laster. (referera Den skapande människan)

\subsection{Medeltiden}    
Under medeltiden lyckades människan ta naturens krafter i anspråk på nytt sätt med en ny uppfinning - vattenkvarnen. Dessa kunde användas för att göra det mesta, till exempel driva sågverk och maskiner. Nu fanns en tendens till att människan började separeras från kraftutövningen. (referera föreläsning medeltiden) 
\newline
\newline
I Europa kunde en stor befolkningstillväxt ses på grund av överskottet på mat från det effektiviserade jordbruket. (referera föreläsning medeltiden)
\newline
\newline
Bättre segelfartyg konstruerades som följd av den ökande handelskonkurrensen mellan nationer. Dessa ledde till längre och ökade transporter. (referera Den skapande människan)
 
