\section{Historiskt perspektiv}
Nedan följer viktiga uppfinningar och ändringar människor gjort från stenåldern fram tills idag. Dessa fakta kommer jag referera till i min diskussion kring frågeställningarna. 
 
\subsection{Stenåldern}
Under stenåldern var människans främsta födokälla storvilt och för att kunna jaga dessa ställdes krav på människans redskap och verktyg. Tillverkningsmetoder utvecklades för att ge mer effektiva och dödliga vapen med skarpare egg. Olika fångsteranordningar och fällor för att fånga och döda djuren utvecklades också. Med skivtekniken, där avslagen används, kunde redskap med mycket god skärpa tillverkas som sågliknande redskap. Under senare tid tillkom även mer avancerade vapen som spjut och yxor.  
\newline
\newline
Människorna hade även god kännedom om djur som får, getter, hästar och kor vilket tyder på en tidig grund för boskapsskötsel. 
\newline
(referera Den skapande människan)
     
\subsection{Efter istiden}
Efter istiden blev det en minskad tillgång på storvilt, dels på grund av klimatförändringarna som följde, dels på grund av människans effektiva jakt. Människan blev nu tvungen att slå sig ned och skapa mer permanenta bosättningar. De främsta näringstillgångarna blev nu istället livsmedelsproduktion från jordbruk och boskapskötsel. Domesticering (avling) av får, getter, nötboskap och grisar för att bland annat få en större köttproduktion började ske under denna period. 
\newline
\newline
Med en växande betydelse för boskapsskötsel och ett tilltagande befolkningstryck medförde sökande efter nya betesmarker. Kring floderna Eufrat och Tigris finns naturliga betesområden och det var här civilisationens vagga Mesopotamien uppkom.   
\newline
\newline
(referera Den skapande människan)

\subsection{Antiken}
Med ugnar kunde metallen nyttjas av människan, då främst koppar och guld och legeringar som brons. Bronset ersattes senare av järn med gjutjärnstekniken. Vapen och verktyg kunde nu förstärkas och effektiviseras. Ugnen såg också förbättringar som effektivare smältugnar. Med allt djupare gruvor uppfanns primitiva pumpar för att få upp grundvattnet som orsakade översvämningar. (referera skapande människan)
\newline
\newline 
Då det nya samhället var präglat av fast boende ställdes krav på lagring och konservering. (referera Den skapande människan)
\newline
\newline
Gamla konserveringsmetoder för kött var antingen rökning eller soltorkning men nu började även saltet användas som konserveringsmetod. Saltet var så viktigt att produktionen kontrollerades av härskarna. 
(referera the story of food preservation) 
\newline
\newline
Saltet blev en mycket viktig handelsvara. I antikens Kina utvecklades en ny effektiv metod för att utvinna salt för att täcka det ökande behovet - djupborrningstekniken. Det var för dåtiden en mycket modern teknik som inte skulle bli känd i Europa förrän långt senare. (referera Den skapande människan)
\newline
\newline
Det var långt ifrån alla regioner som kunde klara av sin egen försörjning och med en ökande handel ställde detta krav på transporterna. Under antiken fanns det stora fartyg som kunde frakta hundratals ton och med uppfinningen av vattenfast murbruk kunde stora hamnanläggningar till dessa konstrueras. Vägnätet utvecklades också men spelade en förhållandevis liten roll då dåtidens vagnar inte kunte transportera några större laster. (referera Den skapande människan)

\subsection{Medeltiden}    
Under medeltiden lyckades människan ta naturens krafter i anspråk på nytt sätt med en ny uppfinning - vattenkvarnen. Dessa kunde användas för att göra det mesta, till exempel driva sågverk och maskiner. Nu fanns en tendens till att människan började separeras från kraftutövningen. (referera föreläsning medeltiden) 
\newline
\newline
Segelfartygen såg förbättringar som följd av den ökande handelskonkurrensen mellan nationer. Detta ledde till längre och ökade transporter. (referera Den skapande människan)

\subsection{1700-talet}
På 1700-talet skedde ett ökat intresse för avel, ej baserat på någon teoretisk kunskap utan på praktisk, vilket ledde till att man fick fram större och kraftigare slaktdjur. Förbättrad utfodring var också en bidragande faktor. (referera Den skapande människan)
\newline
\newline
En av de tidigaste mekaniserade fabrikerna byggdes i England som nyttjade maskiner och arbetare på ett organiserat sätt. Ett inplanterade av en ny arbetsdisciplin och organisation där arbetarna underordnade sig fabrikslivets villkor ledde vägen till en effektivare industri. (referera Den skapande människan) 
\newline
\newline
Med en ökande befolkningsmängd och urbanisering ökade efterfrågan på kol som användes vid järnproduktionen. Allt djupare gruvor krävdes och de översvämmades ofta av grundvattnet. Detta ställde i sin tur krav på bättre pumpar och lösningen blev då ångmaskinen. 
(referera Industriella revolutionen föreläsning)
\newline
\newline
Ångmaskinen revolutionerade fabriksdriften och banade väg för förbättrade och nya transportmedel som ångfartyg och ånglok. 

\subsection{1800-talet}
Med ångloket och järnvägens uppkomst under 1800-talet kunde råvor och färdiga produkter fraktas i helt andra mängder till lands och förbättrade dessutom kommunikationen mellan samhällena. (referera Den skapande människan)
\newline
\newline
Under 1800-talet tillkom nya tekniker för konservering av mat. Konserveringsburkar, frysar kylda med is och sedan den elektroniskt kompressordrivna frysen möjliggjorde längre förvaring av mat (ref den Skapande människan). Det möjliggjorde också en exportering av mat på en global skala (ref Edgerton). 
\newline
\newline
En viktig förutsättning för massproduktion var löpandebandmetoden som gick ut på att varan förs fram på ett kontinuerligt transportband och varje arbetare utför endast ett visst arbetsmoment. Detta ledde till ett starkt minskat behov av arbetskraft och att den var lätt utbytbar - faktorer som möjliggjorde effektivare, massproducerande företag. (referera den Skapande människan) 
\newline
\newline
Elektricitetens uttnjande samt de olika motorer som utvecklades möjliggjorde en allt rationellare fabriksdrift. (ref Den skapande människan)

\subsection{1900-talet tills idag}
Henry Fords massproducerade bensindrivna bil är svår att inte hitta något område i samhällslivet som den påverkade, kommunikationen på land en av dem. (ref Den skapande människan) 
\newline
\newline
Penicillin utveklades under 1940-talet och användes för att döda bakterier i människor. Det blev även en viktig del i kontrollerandet av sjukdomar i de packade djurpopulationerna i den nya industrialiserade djurhållningen. Man märkte även att kycklingar växte snabbare av penicillin vilket ledde till att det sattes i maten djuren åt. (ref edgerton) 
\newline
\newline
I början av 1900-talet gjordes de första testflygningarna med motorplan. De såg en snabb utveckling i och med de två världskrigen som sedan bäddade för en fortsatt utveckling inom det civila. (ref Den skapande människan) 
\newline
\newline
Gasning. 
\newline
\newline
Informationssamhällets framväxt med radio, telefon, tv och senare internet hjälpte till att utveckla marknaden för företagen. (referera kommunikationshistoria)