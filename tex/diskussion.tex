\section{Diskussion}
Nedan följer en diskussion kring möjliga orsaker för de teknikutvecklingar som nämndes i Resultat. 

\subsection{Teknikutveckling}
Utan den situationen människan hade satt sig i med sin jaktiver efter istiden hade kanske inte människan slagit sig ned och bildat permanenta bosättningar och blivit boskapsskötare. Ett annat exempel gruvdriften. Under antiken när metalltillverkningen blev viktig krävdes allt djupare gruvor för att täcka vårat behov. Detta resulterade i översvämningar vilket krävde en ny uppfinning, pumpen. Senare under 1700-talet behövdes ännu bättre pumpar då efterfrågan på metall blivit betydligt större, resultatet blev då ångmaskinen. Enligt min åsikt drivs alltså ny teknikutveckling av vår påverkan på miljön och vår efterfrågan. Jag drar också parallellen att teknik skapar behovet av ny teknik, till exempel metalltillverkningen skapade pumpar. 
\newline
\newline
Vår efterfråga på olika 
\newline
\newline
Med en ökande handel ställdes krav på transporterna. 
       

\subsection{Miljö- och samhällspåverkan}
